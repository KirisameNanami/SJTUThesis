%# -*- coding: utf-8-unix -*-
%%==================================================
%% chapter01.tex for SJTU Master Thesis
%%==================================================

%\bibliographystyle{sjtu2}%[此处用于每章都生产参考文献]
\chapter{总结与讨论}
\label{chap:Con}
在本轮文中,我们给出了一个新的算法来解决黄佳琳提出的AKI问题,其不仅可以用来分析和确定密钥编排方案的扩散程度,也可以用来通过减少猜测集合的大小来优化如相关密钥攻击的密码攻击。
同事,我们也给出了一个新的方法来优化与重新设计一个没有密钥信息泄露的密钥编排方案。
为了展示我们提出的新方法的可行性和有效性,我们优化了对23轮TWINE-80的零相关攻击,并提出了一个新的对12轮RECTANGLE-128的中间相遇攻击,同时分别给出了新的密钥编排方案来帮助其抵抗这些攻击。

密钥依赖路径由于其与密钥编排方案无关,只与计算路径(轮函数)有关的特性,成为了使用AKI来为轮函数分析或设计密钥编排方案的重要途径。
在轮函数和轮密钥选择确定的前提下,某轮的某一比特的密钥依赖路径即随之确定,而此时不同的密钥编排方案就会带来不同的AKI,形成了物理实验中常见的“控制变量法”。
因此,要设计一个与轮函数相匹配的密钥编排方案,我们要做的就是找到一个AKI达到理论最大值的密钥编排方案。
为此,我们首先就需要一个能计算出精确AKI数值的算法(一个只能算出上界的算法只能用于攻击中优化猜测集合,而不能用来衡量编排方案的优劣)。
本文中使用了以密钥编排方案为基础构建流量网络图的方法,使用最大流问题解决了AKI问题。
进而我们需要反向思考,如何以“达到AKI理论最大值”为前提条件,找到符合条件的一个密钥编排方案。
本文中给出了一个对一个特定的子密钥集合设计一个相对较好的密钥编排方案的方法,但是其限制在于只能对一个集合达到AKI理论最大值,同时也无法保证是一个最优最简化的编排方案。
为了为轮函数设计一个更加普适的,能够对TKI超过主密钥长度的任何猜测集合都达到AKI理论最大值的密钥编排方案,我们就需要找到一个对不同的密钥依赖路径都有AKI最大的编排方案。
从单个集合设计的方案到能适应所有集合的方案的扩展十分困难,特别是当两个集合所设计出的方案大相径庭时,想要合并起来还要保持方案的简单性更加困难。
本文所提到的RECTANGLE-128的密钥编排设计主要是由于其依赖路径的特殊性(使所有依赖路径的AKI都达到最大的矩阵恰巧都为一个移位后的单位矩阵)才得到了一个普适的编排方案。
想要得到更加普遍的确定编排方案的方法,还需要进一步考虑其依赖矩阵的确定方式。

尽管,以我们有限的知识,我们只能给出一个基于给定的子密钥集合来寻找一个可能的、较小的密钥依赖矩阵的方法,我们仍然相信我们所构建的密码学与图论的“桥梁”将会帮助我们总结出一系列的科学的、理论的、完备的密钥编排方案设计方法与原则。
