%# -*- coding: utf-8-unix -*-
%%==================================================
%% chapter01.tex for SJTU Master Thesis
%%==================================================

%\bibliographystyle{sjtu2}%[此处用于每章都生产参考文献]
\chapter{密钥编排方案的设计}
\label{chap:Design}
在前几章的讨论中,我们将密钥编排方案通过算法\ref{alg:const}量化成了一个流量网络。
在此基础上,我们可以进一步考虑如何去设计一个密钥编排方案,让它在密钥信息这一方面上表现得更好。
理论上,对于一个主密钥长度为$n$的密码函数,给定一个子密钥集合$K$,如果$|K|\geq n$,则必然存在一个密钥编排方案能够让$K$的AKI达到$n$。
为了更普遍的描述一个子密钥集合$K$包含密钥信息的理论最大值,我们给出如下定义:
\begin{defn}[理论密钥信息(TKI)]
    对于主密钥长度为$n$的密钥编排方案,一个子密钥集合$K$的理论密钥信息(TKI)为其包含密钥信息的理论最大值,为$min(|K|,n)$。
\end{defn}
TKI实际上是AKI的一个上界,是密钥编排方案完全扩散(即其依赖矩阵为全1矩阵)时$K$的AKI,但是全扩散的密钥编排方案并不能在实际中得到运用。
因此,我们需要一个尽可能小的(1的数量尽可能少的)依赖矩阵来达到实际运用的目的。

\section{单个子密钥集合下的矩阵设计}
给定一个子密钥集合$K$,设计一个尽可能小的依赖矩阵$M$使得集合$K$的AKI达到$n$。
子密钥集合猜测的技术在许多攻击中都有设计,我们在第\ref{chap:App}章中讨论了两种设计密钥猜测的攻击方式。
在这一节中,我们将给出一个贪心算法来输出一个相对较小的矩阵$M$。

简单来说,在我们的贪心算法算法\ref{alg:Matrix}中,我们在由$K$和全1矩阵$M_{full}$通过算法\ref{alg:const}所建立起来的流量网络$G_f$的基础上,给每一条弧都新添加一个参数$cost$表示费用。
除了像$(u_{out},v_{in})$这样的代表前后轮比特依赖关系的弧的费用是1以外,别的所有弧的费用均为0.
然后,我们计算出流量网络$G_f$的一个最小费用最大流\footnote{最小费用最大流值的是在有费用的流量网络$G_f$中所有最大流中费用最小的流,其中流的费用被定义为$\sum f(e)\cdot cost(e)$}后,将该流中所有涉及到的弧提取出来形成一个新的矩阵$M$(可以看成将所有弧压缩到一轮中)。
由于$G_f$是由全扩散的密钥编排方案构造出来的,我们计算出的最小费用最大流$f$的流量显然是$n$;
又注意到如果我们重新由$K$和$M$构造出一个新的流量网络$G'_f$,上述的最小费用最大流$f$在该网络中依然成立,因为所有$f$涉及的弧都被存在了$M$中。
因此,该矩阵$M$将会使$K$的AKI为$n$。
同时,由于我们对该流“费用最低”的要求,$f$保证了涉及尽可能少的依赖关系(但没有考虑在不同轮中共用同一种依赖关系的可能性),因此$M$将会是一个符合条件的相对较小的矩阵(但不一定是最小的)。
\begin{algorithm}
    \caption{矩阵设计算法}
    \begin{algorithmic}[1]
        \Function{GetMatrix}{$K$}
            \State $G \gets$\Call{Construct}{$K,M_{full}$}
            \ForAll{$e\in G.E$}
            \State $cost(e)\gets 0$
            \EndFor
            \ForAll{$(u_{out},v_{in})\in G.E$}
            \State $cost((u_{out},v_{in}))\gets 1$
            \EndFor
            \State $f \gets$\Call{Get\_Min\_Cost\_Max\_Flow}{$G(E,V,c),cost$}
            \State $M \gets \mathbf{0}$
            \ForAll {$f((u_{out},v_{in}))>0$}
            \State $M[u.bit][v.bit]\gets 1$
            \EndFor
            \State \Return $M$
        \EndFunction
    \end{algorithmic}
    \label{alg:Matrix}
\end{algorithm}

注意到$M$可以是密钥编排方案的一个中间矩阵,也就是说你可以为了更方便的以$M$为基础设计编排方案,在$M$中添加额外的1且保持AKI为$n$。

\noindent
\textbf{应用于对TWINE-80的攻击:}我们在第\ref{chap:App}章中提到了一个使用$((2,9),4,14,5)$区分器的对TWINE-80的23轮零相关攻击。
在这个攻击中,我们使用了TWINE-80密钥编排方案的密钥信息泄露,将原本由19密钥块大小的猜测子密钥集合$K$降低到了16子密钥块的大小。
为了让这个优化变得不可能,我们需要重新设计一个密钥编排方案来让$K$的AKI达到理论上最大(19个子密钥块)。
由算法\ref{alg:Matrix},我们找到了如下矩阵:
$$\left(
    \begin{array}{cccccccccccccccccccc}
0&0&0&0&0&0&0&0&0&0&0&0&0&0&0&0&0&0&0&0\\
0&0&0&0&0&0&0&0&0&0&0&0&0&0&0&0&0&1&0&0\\
0&0&0&0&0&0&0&0&0&0&0&0&0&0&0&0&0&0&0&0\\
0&0&0&0&0&0&0&0&0&0&0&0&0&0&0&0&0&0&1&0\\
0&1&0&0&0&0&0&0&0&0&0&0&0&0&0&1&0&0&0&0\\
0&0&1&0&0&0&0&0&0&0&0&0&0&0&0&0&0&0&0&0\\
0&0&0&0&0&1&0&0&0&0&0&0&0&0&0&0&1&0&0&1\\
0&0&0&0&1&0&0&1&0&0&0&0&0&0&0&0&0&0&0&0\\
0&0&0&0&0&1&0&1&1&0&0&0&0&0&0&0&0&0&0&0\\
0&0&0&0&0&0&1&0&1&1&0&0&0&0&0&0&0&0&0&0\\
0&0&0&0&0&0&0&1&0&1&1&0&0&0&0&0&0&0&0&0\\
0&0&0&0&0&0&0&0&1&1&1&1&0&0&0&0&0&0&0&0\\
0&0&0&0&0&0&0&0&0&1&1&1&1&0&0&0&0&0&0&0\\
0&0&0&0&0&0&0&0&0&0&1&1&1&1&0&0&0&1&1&1\\
0&0&0&0&0&0&0&0&0&0&0&1&1&1&1&0&0&0&1&1\\
0&0&0&0&0&0&0&0&0&0&0&1&1&1&1&1&0&0&0&1\\
0&0&0&0&0&0&0&0&0&0&0&0&1&0&1&1&1&0&0&1\\
0&0&0&0&0&0&0&0&0&0&0&0&0&1&1&1&1&1&0&0\\
0&0&0&0&0&0&0&0&0&0&0&0&0&0&1&0&1&1&1&0\\
0&0&0&0&0&0&0&0&0&0&0&0&0&0&0&1&0&1&1&1\\
\end{array}
\right)
$$

\section{为轮函数设计密钥编排方案}
黄佳琳博士在\citen{huang2014revisiting}提出了利用AKI来评估密码算法的密钥编排方案的方法,即分析一轮中所有比特的密钥依赖路径的实际密钥信息,进而用该轮最低的AKI来确定密钥编排方案经过该轮次后是否存在密钥信息泄露。
理论上,一旦密钥依赖路径的大小超过了主密钥长度$n$,该轮的实际密钥信息就可能能够达到$n$。
注意到密钥依赖路径的大小只与计算路径有关,而计算路径只依赖于轮函数,因此在给定轮函数的情况下,我们就可以确定从哪一轮开始实际密钥信息能够在理论上达到最大值$n$。
因此,寻找一个理想的密钥编排方案就相当于寻找一个尽可能小的依赖矩阵$M$使得第$r_0$轮(该轮的密钥依赖路径大小开始超过$n$)的实际密钥信息达到$n$。
但由于一整轮的实际密钥信息是该轮所有密钥依赖路径的实际密钥信息的最小值,要寻找一个矩阵$M$使得所有密钥依赖路径$K_1,K_2,\dots$的AKI都达到$n$是一个十分困难的问题,我们暂时无法想到一个有效的计算方法。
但我们仍然相信我们构建出的密钥编排方案与流量网络的“桥梁”将会帮助我们找到密钥编排方案的设计原则。

\section{一个RECTANGLE-128的优化密钥编排方案}
我们计算了RECTANGLE-128的实际密钥信息,找到了它的密钥编排方案存在着比较严重的密钥信息泄露。
这个泄露所导致的中间相遇攻击在第\ref{chap:App}章中已有所讨论。表\ref{tab:Rec}中展示了其具体的AKI值。
\begin{table}[htbp]
    \centering
    \begin{tabular}{c|c|c|c|c}
        \multirow{2}{*}{轮数} & \multicolumn{2}{c|}{RECTANGLE-128} & 简单移位方案 & 优化后的编排方案\\
        \cline{2-5}
        & $TKI$ & $AKI$ & $AKI$ &$ AKI $\\
        \hline
        $r=1$ & 4 & 4 & 4 & 4\\
        $r=2$ & 20 & 18 & 20 & 20\\
        $r=3$ & 56 & 44 & 52 & 53\\ 
        $r=4$ & 120 & 83 & 100 & 102\\ 
        $r=5$ & 128 & 103 & \textbf{128} &\textbf{128}\\ 
        $r=6$ & 128 & 120 & 128 & 128\\ 
        $r=7$ & 128 & \textbf{128} & 128& 128\\ 
        \hline
    \end{tabular}
    \bicaption[tab:Rec]{RECTANGLE-128在不同密钥编排方案下的实际密钥信息}{RECTANGLE-128在其原编排方案、简单移位方案和优化方案下的实际密钥信息}{AKI of RECTANGLE-128}{The AKI of the RECTANGLE-128 schedule and the optimized schedule}
\end{table}
\begin{algorithm}
    \caption{RECTANGLE-128原密钥编排方案的密钥扩展函数}
    \begin{algorithmic}[1]
        \Function{OriginalScheduleFunc}{$WK_{i-1}$}
        \State $WK_i \gets$ \Call{SubCell}{$WK_{i-1}$}
        \State $WK_i \gets$ \Call{FeistelTransform}{$WK_{i}$}
        \State $WK_i \gets$ \Call{AddConstant}{$WK_{i}$}
        \State \Return $WK_i$
        \EndFunction
        \Function{OriginalKeyExtract}{$WK_i$}
        \State $RK_i \gets$ \Call{GetRight16Columns}{$WK_i$}
        \State \Return $RK_i$
        \EndFunction
    \end{algorithmic}
    \label{alg:key}
\end{algorithm}

为了优化RECTANGLE-128的密钥编排方案进而使得其能够抵抗第\ref{chap:App}中提到的中间相遇攻击,我们保留了RECTANGLE-128的密钥提取方案(轮密钥由子密钥的最右侧16列构成),使用算法\ref{alg:Matrix}来寻找一个能够使$K_t$和$K_b$的AKI达到128的矩阵。
惊讶的是,两个子密钥集合所对应的矩阵均为一个循环向右移位16比特的单位矩阵的一部分。
我们进一步分析了RECTANGLE-128第5轮的所有密钥依赖路径(第5轮的TKI已经达到128),而这些依赖路径在算法\ref{alg:Matrix}中给出的矩阵均为上述移位后的单位矩阵。
因此,我们有如下观察结论:

\noindent
\textbf{观察结论:}一个子密钥生成是简单移位($WK_i=WK_{i-1}\ll 16$)且密钥提取方案与RECTANGLE-128原提取方案相同(提取右侧16列)的简单移位密钥编排方案是RECTANGLE-128的一个理想密钥编排方案。(表\ref{tab:Rec}中列出了其AKI值)

注意到RECTANGLE-128的原密钥编排方案的依赖矩阵中存在着一条右移32位后的对角线1(即移位后的单位矩阵),我们可以在尽量不破坏设计者最初的设计理念的情况下,在原RECTANGLE-128的编排方案上稍加改动使其符合上述理想编排方案。
我们在优化方案(算法\ref{alg:opt})中添加一个新的“移位”操作(行\ref{alg:shift})。
表\ref{tab:Rec}中列出了优化方案的AKI值。
\begin{algorithm}
    \caption{优化后的密钥编排方案}
    \begin{algorithmic}[1]
        \Function{KeyExpand}{$K$}
        \State $WK_0 \gets K$
        \For {$i\in 1\dots 25$}
        \State $WK_i \gets$ \Call{OriginalScheduleFunc}{$WK_{i-1}$}
        \State \textbf{$WK_i \gets WK_i \ll 16$} \label{alg:shift} \Comment{新增的移位操作}
        \State $RK_i \gets$ \Call{OriginalKeyExtract}{$WK_i$}
        \EndFor
        \State \Return $WK$
        \EndFunction
    \end{algorithmic}
    \label{alg:opt}
\end{algorithm}

由于我们的优化方案能够使RECTANGLE-128从第5轮开始AKI就达到理论最大值128,第\ref{chap:App}中提到的中间相遇攻击就会因为无法找到存在密钥信息泄露的6轮路径(甚至无法找到5轮路径)而变得不可能。
并且,该优化方案可以将基于密钥信息泄露的中间相遇攻击的轮数限制在8轮(4+4),这意味着我们的优化密钥编排方案将会比RECTANGLE-128的原编排方案提供更高的安全性保证。
