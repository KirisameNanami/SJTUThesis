%# -*- coding: utf-8-unix -*-
\begin{bigabstract}
The low diffusion of a simplified key schedule in block ciphers is usually responsible for many attacks.
    Huang \emph{et al.} gave conceptions on Key Information, especially the Actual Key Information (AKI), which successfully illustrated and quantified the method to evaluate the diffusion of a key schedule.
    However, the algorithm used to calculate AKI proposed by Huang \emph{et al.} cannot be used to determine the weakness or further make an optimization of a key schedule since it cannot give an accurate value of AKI.
    In this paper, we successfully solve the AKI problem by our AKI - Minimum Cut Algorithm based on the well-known Max-flow problem in Graph Theory,
    and further give a method to design a key schedule which offers better security with less or without Key Information leakage.
    As applications, we optimize the zero-correlation attack on 23-round TWINE-80 and find a new meet-in-the-middle attack on 12-round RECTANGLE-128,
    and respectively give a new dependency matrix for TWINE-80 and an optimized key schedule for RECTANGLE-128, which makes both of them stronger against these attacks.

A key schedule is an algorithm that expands a relatively short master key to a large expanded key as round keys used in encryption and decryption algorithms.
When it comes to block ciphers, the key schedules are often simplified due to implementation considerations and cause weakness that can be exploited in many attacks, especially for lightweight block ciphers since their schedule are designed based on not only the security but also the effciency of software and hardware.
Key schedules like that of PRESENT\citen{Bogdanov}, RECTANGLE\citen{zhang2015rectangle}, TWINE\citen{Suzaki_2013} and LBlock\citen{Wu_2011} are round-by-round iterations with low diffusion;
Key schedules like that of SIMON and SPECK\citen{beaulieusimon} have simple permutations or linear operations with low diffusion.
Some ciphers like LED\citen{Guo2011} do not even have a key schedule, and just use master keys directly in each round.
Those key schedules are simple and efficient but responsible for many attacks, especially related-key attack\citen{biham1994new,Biryukov2009,Ko2004} and meet-in-the-middle attack\citen{diffie1977special,Biryukov2015,Bogdanov2011}.

In order to illustrate and quantify the diffusion and the detailed weakness which should be responsible for the genetic attacks, Huang \emph{et al.} gave conceptions on Key Information in \citen{huang2014revisiting}, including Computation Path, Key Dependency Path and Actual Key Information, which vividly described the diffusion of a key schedule. 
Taking use of the Key Information leakage of key schedules, Huang \emph{et al.} derived meet-in-the-middle attacks on 40-round SHACAL-2\citen{Handschuh2002} and 25-round XTEA\citen{Needham1997} in \citen{huang2014revisiting} and further proposed an algorithm in \citen{Huang_2014} which could give a reasonable upper bound of the AKI of a key schedule.
However, to design a key schedule which offers more security in Key Information, an accurate value of AKI is required and a method or principle for such a design is essential.

\noindent
\textbf{Our Contribution}. In this paper, we demonstrate a polynomial algorithm solving the AKI problem based on Graph Theory, building a "bridge" between key schedules and flow networks.
We describe the method solving AKI problem from the highly simplied version problem of 1-round AKI, and prove the equivalence between 1-round schedule and bipartite graph.
Further, we introduce the Max-flow Min-cut Theorem and give our AKI - Mimimum Cut Algorithm.
A theoretical proof of the correctness of our algorithm is also included.
By calculating the accurate AKI, we optimize the zero-correlation attack on 23-round TWINE-80 and derive a new meet-in-the-middle attack on 12-round RECTANGLE-128.

Meanwhile, taking use of the "bridge" between key schedules and flow networks we build, we give a greedy algorithm which can generate a new schedule (presented in dependency matrix) from a given subkey set without Key Information leakage.
Based on this algorithm, we respectively give a new matrix for TWINE-80 against the zero-correlation attack and a new key schedule for RECTANGLE-128 which makes any meet-in-the-middle attack on more than 8 rounds impossible.

The source code of the algorithms mentioned above is available \href{https://github.com/KirisameNanami/AKI-Algorithms}{here}.

In order to determine the existance of Key Information leakage, there are several algorithms to compute AKI.
Huang has given an automatic detection tool which includes the calculation of AKI in \citen{Huang_2014}, but the algorithm was a greedy algorithm, which means that it could only give a possible upper bound of AKI.
Also, the Automatic Search of Key-Bridging Technique\citen{dunkelman2010improved} proposed by Lin \emph{et al.} in \citen{lin2016automatic} could also find an upper bound of AKI (since a key bridge is exactly 1-bit leakage of Key Information).
However, neither of the two algorithm could compute the accurate value of AKI (a counterexample will be provided in following sections), which made them a tool for generating an attack such as meet-in-the-middle attack \citen{diffie1977special}, but not for evaluating the safety of a key schedule.
Also, the low efficiency of both of the algorithms (usually several minutes for a single subkey set) limit the application to a whole key schedule, since we have to examine the KDPs from all the internal bits.
In this paper, we will demonstrate our AKI - Minimum Cut Algorithm based on Graph Theory which can get the accurate value of AKI.
This AKI algorithm is based on graph theory, aims at transform the key schedule (represented by the dependency matrix $M$) and a subkey bits set $K$ to a directed flow graph $G_f(E,V,c)$.
Algorithm .~\ref{alg:const} is used to construct such graph.
Briefly speaking, besides the two completely new vertex $s$ and $t$, we construct two vertex $u_{in},u_{out}$ for each bit which is or is depended by the bits in $K$ according to the key schedule $M$, and connect them from $u_{in}$ to $u_{out}$ with a capacity of 1.
All the other edge has its capacity of infinity, and there're three kinds of those edges:
\begin{enumerate}
    \item Edge from $s$ to $u_{in}$ where $u$ is in first round,
    \item Edge from $u_{out}$ to $t$ where $u\in K$,
    \item Edge from $u^{b}_{out}$ to $u^{b'}_{in}$ where $b$ and $b'$ are in two consecutive rounds and $b'$ depends on $b$ according to the key schedule.
\end{enumerate}

For a subkey set $K$ and a dependency matrix $M$ of a $n$-bit key schedule, let $R$ be the maximum round on which bits in $K$ are.
Then the graph $G_f$ constructed by Algorithm.~\ref{alg:const} will have at most $2nR$ vertices (each bit has two vertices and there are at most $R\cdot n$ bits) and $n+|K|+nR+n^2R$ edges (consider the situation that $M$ is a full matrix).
Currently there are many algorithms to solve Max-flow problem. The Push Relabel Algorithm (with highest label node selection rule) has its performance of $\mathcal{O}(V^2\sqrt{E})$ in time complexity, where $V$ and $E$ stands for the number of vertices and edges, respectively.
Hence, the time complexity for our AKI - Minimum Cut algorithm will be $\mathcal{O}((2nR)^2\sqrt{n+|K|+nR+n^2R})=\mathcal{O}(n^3R^{2.5})$, which is polynomial.
It guarantees that for most encryption methods, our algorithm can terminate in seconds.

Since the AKI-set plays the same role with the subkey set $K$ in Key Information, once there exists a leakage of Key Information, the complexity of an attack based on a guessing set $K$ (usually with a distinguisher whose input/output is the set which $K$ is generated from) will be easily reduced by replacing $K$ by its AKI-set $K'_0$.
In this paper, we will introduce two applications based on Key Information leakage.

After we quantify the key schedule with a flow network constructed by Algorithm.~\ref{alg:const}, we can further consider how to design a key schedule with the best performance in terms of AKI.
Theoretically, for a key schedule with a $n$-bit master key and a subkey set $K$, if $|K|\geq n$, there must exist a key schedule with its dependency matrix $M$ which makes the AKI of $K$ be $n$.
This conclusion is trivial since a full matrix (all the cells are 1) $M_{full}$ obviously meet the requirments.
However, such a full matrix $M_{full}$ cannot be used in practical as a dependency matrix of a key schedule.
Hence, we need to find a matrix with the number of '1's as small as possible.

In this paper, we give a new algorithm to give the accurate value of AKI defined by Huang \emph{et al.}, which can be used not only to evaluate and determine the diffusion of a key schedule, but also to optimize the subkey set used in attacks like related-key attack.
We also gives a new method to optimize and re-design a key schedule without Key Information leakage.
To demonstrate the usefulness of those methods, we optimize the zero-correlation attack on 23-round TWINE-80 and generate a new meet-in-the-middle attack on 12-round RECTANGLE-128,
and respectively give new key schedules for both of them which make them secure enough against their attacks.

Although, to the best of our knowledge, we only provide the method of finding a possible, relatively small dependency matrix on a certain subkey set,
we still believe that the "bridge" we built between key schedules and flow networks will work further in summarizing a scientific, theoretical and complete serial of principles on designing the key schedule for a round function.

\end{bigabstract}
