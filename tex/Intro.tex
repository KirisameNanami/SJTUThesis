%# -*- coding: utf-8-unix -*-
%%==================================================
%% chapter01.tex for SJTU Master Thesis
%%==================================================

%\bibliographystyle{sjtu2}%[此处用于每章都生产参考文献]
\chapter{研究现状与本文概述}
\label{chap:Intro}
\section{研究现状}
密钥编排方案是在加密或是解密中都会用到的一种将较短的主密钥扩展成很长的扩展密钥(并用来做轮密钥)的算法。
在分组密码中,考虑到算法的设计与实现,其密钥编排方案往往会比较简单,而过于简单的编排方案往往会导致一些攻击。
在轻量级分组密码中,由于其不仅需要考虑安全性还需要考虑软硬件上的高效与简便的特殊性,这种情况更加普遍。
有些密钥编排方案是扩散程度较低的轮迭代算法,例如PRESENT\citen{Bogdanov},RECTANGLE\citen{zhang2015rectangle},TWINE\citen{Suzaki_2013}和LBlock\citen{Wu_2011};
有些密钥编排方案是简单的置换与线性操作,例如SIMON和SPECK\citen{beaulieusimon};
有些甚至根本没有编排方案,只是简单的在每轮中使用主密钥,例如LED\citen{Guo2011}。
这些密钥编排方案被高度简化,在实现上十分高效简便,但却导致了以相关密钥攻击\citen{biham1994new,Biryukov2009,Ko2004}和中间相遇攻击\citen{diffie1977special,Biryukov2015,Bogdanov2011}为主的攻击。

为了描述和量化密钥编排方案的扩散程度以及其导致攻击的具体的弱点所在,黄佳琳博士等人在\citen{huang2014revisiting}中提出了围绕“密钥信息”的一些概念,如计算路径、密钥依赖路径和实际密钥信息,这些概念形象生动地描述了一个密钥编排方案的扩散程度。
利用密钥编排方案中的密钥信息泄露,黄佳琳等人在\citen{huang2014revisiting}中提出了对40轮SHACAL-2密码\citen{Handschuh2002}和25轮XTEA密码\citen{Needham1997}的中间相遇攻击,并进一步在\citen{Huang_2014}中提出了一个能够给出一个编排方案的AKI合理上界的贪心算法。
然而,为了设计出一个能够在密钥信息上提供更好的安全性的密钥编排方案,我们需要AKI的精确值并从而总结出一个设计方案或原理。

\section{本文主要贡献}
在本文中,我们将会展示我们设计一种基于图论来解决AKI问题的多项式复杂度的算法,建立了密码学与图论之间的“桥梁”。
为了更完整的描述我们解决AKI问题的方法,我们将会从最简单的单轮AKI问题着手,并证明单轮编排方案与二分图的等价性;
之后我们将进一步介绍图论中著名的最大流最小割定理,并由此给出我们设计的AKI-最小割算法。
同时我们也对我们算法的正确性进行了理论证明。
通过计算出AKI的真正数值,我们成功地优化了针对23轮TWINE-80密码的零相关攻击,还设计了一种新的针对12轮RECTANGLE-128密码的中间相遇攻击。

同时,利用我们构建的密钥编排方案与流量网络之间的“桥梁”,我们也给出了一种能够构造出一种针对给定的子密钥集合没有信息泄露的,以依赖矩阵的形式表示的新的密钥编排方案的贪心算法。
基于这种算法,我们分别针对上述两个攻击,为TWINE-80设计了能够抵抗其零相关攻击的编排方案,也为RECTANGLE-128设计了能够抵抗所有8轮以上中间相遇攻击的新密钥编排方案。

关于上述算法的更多信息,请访问在\href{https://github.com/KirisameNanami/AKI-Algorithms}{Github}上的源码。

\section{本文结构安排}
本文的研究对象为分组密码的密钥编排方案,而研究贡献主要围绕AKI-最小割算法、单密钥集合的依赖矩阵设计算法及两者的应用。
本文的后续章节内容安排如下:

第三章为后续章节预备关于密钥信息的一系列概念与基本知识,并使用了黄佳琳在\citen{huang2014revisiting}中提出的玩具密码的例子来帮助理解;
并在黄佳琳对实际密钥信息(AKI)的定义上做略微扩展,使其应用范围更加普适,能够更广泛地应用在各种攻击中常见的密钥猜测集合中。

第四章中我们将介绍AKI-最小割算法来解决AKI问题。首先我们将AKI问题简化成仅有单轮加密情况下的子问题,描述并证明了该情况下密钥编排方案模型与二分图的等价性,并使用二分图最大匹配来解决这个子问题;
之后我们在介绍一个典型反例的帮助下,引出更加普遍情况下的AKI问题的解决方法。
我们将简单介绍图论中关于流量网络中流与割的定义,并介绍著名的最大流-最小割定理,为AKI-最小割算法的介绍与证明做准备;
随后我们将会介绍AKI-最小割算法的构图方法以及该流量网络与密钥编排方案之间的紧密关系;
为了证明AKI-最小割算法的正确性,我们使用最大流-最小割定理,从两个方向进行了证明,最终证明最小割与实际密钥信息集合的等价性。
最后,我们分析了AKI-最小割的时间复杂度,理论说明了该算法的高效性与实用性。

之后的第五章中,我们针对密钥信息泄露所导致的攻击列出了两个使用AKI-最小割算法的应用。
首先,我们对Wang和Lin等人提出的对23轮TWINE-80的两个零相关攻击进行了改正与优化,降低了猜测密钥集合的复杂度;
然后,我们提出了一个对12轮RECTANGLE-128的新的中间相遇攻击,由于这一类攻击可以十分简单的由AKI-最小割算法得出的结果得出,该攻击也侧面证明了密钥信息泄露的严重性。

在讨论了密钥信息泄露的攻击之后,我们转而在第六章中讨论如何设计密钥编排方案来防止密钥信息泄露。
在该章中我们将介绍一个对单个子密钥集合设计无密钥信息泄露的密钥编排方案的算法,利用最小费用最大流得出一个较小的密钥依赖矩阵;
使用该算法,我们对第五章中提到的TWINE的一个零相关区分器的密钥猜测部分进行了分析,并设计了一个新的密钥编排方案依赖矩阵使得该区分器的密钥猜测部分无法被优化,从而使该攻击的时间复杂度上升。
之后,我们介绍了如何为一个特定的轮函数设计无密钥信息泄露的密钥编排方案,并为一个特殊的例子——RECTANGLE-128设计了一个优化的密钥编排方案,
该密钥编排方案能够在理论上最少的轮数使AKI达到理论最大值,从而能够抵抗任意8轮以上的中间相遇攻击。

第七章中,我们将会介绍一系列相关工作,主要包含了使用AKI-最小割算法对各种不同分组密码算法的AKI分析。
首先,我们对两个十分相似的SPN结构轻量级分组密码PRESENT和RECTANGLE各自的两种密钥编排方案进行了分析,并对两者不同的结果的原因进行了讨论。
然后我们对两个拥有极其简单的密钥编排方案的密码算法Midori和LED进行了AKI分析,并将其与PRESENT和RECTANGLE相比较,讨论密钥编排方案与轮函数之间的紧密关系。
最后,在讨论了四种SPN结构密码函数后,为了说明Feistel结构的密码算法同样适用于我们的AKI-最小割算法,我们用SIMON这一经典的Feistel结构密码算法为例,详细说明了如何用AKI-最小割算法来分析Feistel结构密码函数,其中包含如何确定非简单迭代使用密钥扩展函数的密钥编排方案的依赖矩阵,以及Feistel结构密码的轮AKI的确定方法讨论。
